\documentclass[DM,authoryear,toc]{lsstdoc}
% lsstdoc documentation: https://lsst-texmf.lsst.io/lsstdoc.html
\input{meta}

% Package imports go here.

% Local commands go here.

%If you want glossaries
%\input{aglossary.tex}
%\makeglossaries

\title{Measurement of Faint DIASources in LSST Prompt Processing}

% Optional subtitle
% \setDocSubtitle{A subtitle}

\author{%
Eric Bellm, Melissa Graham, Leanne Guy and the DM System Science Team
}

\setDocRef{DMTN-228}
\setDocUpstreamLocation{\url{https://github.com/lsst-dm/dmtn-228}}

\date{\vcsDate}

% Optional: name of the document's curator
% \setDocCurator{The Curator of this Document}

\setDocAbstract{%
The DMS will have the ability to detect and measure a limited number of DIASources  below the nominal 5$\sigma$ detection threshold that satisfy additional criteria. 
}

% Change history defined here.
% Order: oldest first.
% Fields: VERSION, DATE, DESCRIPTION, OWNER NAME.
% See LPM-51 for version number policy.
\setDocChangeRecord{%
  \addtohist{1}{2022-XX-XX}{Unreleased.}{Leanne Guy}
}


\begin{document}

% Create the title page.
\maketitle
% Frequently for a technote we do not want a title page  uncomment this to remove the title page and changelog.
% use \mkshorttitle to remove the extra pages

% Content
\section{Introduction}\label{sec:into}

% Leanne - introduce subject  and associated with requirements 
\section{Use Cases}\label{sec:usecases}

Here we present the motivating use cases for the measurement of sub-threshold DIASources.

\subsection{Quality Assessment}\label{ssec:qa}
% Eric 

\subsection{Strong Lensing}\label{ssec:sl}
% Leanne

\subsection{Target of Opportunity}\label{ssec:too}
% Eric

\subsection{Potentially Hazardous Asteroids}\label{ssec:pha}
% Mario
\section{Technical Consideration}\label{sec:technical}
% Eric
\section{Community Proposal Process}\label{sec:process}
% Melissa
\section{Timeline}\label{sec:timeline}
% Leanne


\appendix
% Include all the relevant bib files.
% https://lsst-texmf.lsst.io/lsstdoc.html#bibliographies
\section{References} \label{sec:bib}
\renewcommand{\refname}{} % Suppress default Bibliography section
\bibliography{local,lsst,lsst-dm,refs_ads,refs,books}

% Make sure lsst-texmf/bin/generateAcronyms.py is in your path
\section{Acronyms} \label{sec:acronyms}
\addtocounter{table}{-1}
\begin{longtable}{p{0.145\textwidth}p{0.8\textwidth}}\hline
\textbf{Acronym} & \textbf{Description}  \\\hline

DM & Data Management \\\hline
DMS & Data Management Subsystem \\\hline
DMTN & DM Technical Note \\\hline
LSST & Legacy Survey of Space and Time (formerly Large Synoptic Survey Telescope) \\\hline
\end{longtable}

% If you want glossary uncomment below -- comment out the two lines above
%\printglossaries





\end{document}

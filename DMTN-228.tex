\documentclass[DM,authoryear,toc]{lsstdoc}
% lsstdoc documentation: https://lsst-texmf.lsst.io/lsstdoc.html
\input{meta}

% Package imports go here.

% Local commands go here.

%If you want glossaries
%\input{aglossary.tex}
%\makeglossaries

\title{Measurement of Faint DIASources in LSST Prompt Processing}

% Optional subtitle
% \setDocSubtitle{A subtitle}

\author{%
Eric Bellm, Melissa Graham, Leanne Guy and the DM System Science Team
}

\setDocRef{DMTN-228}
\setDocUpstreamLocation{\url{https://github.com/lsst-dm/dmtn-228}}

\date{\vcsDate}

% Optional: name of the document's curator
% \setDocCurator{The Curator of this Document}

\setDocAbstract{%
The DMS will have the ability to detect and measure a limited number of DIASources  below the nominal 5$\sigma$ detection threshold that satisfy additional criteria. 
}

% Change history defined here.
% Order: oldest first.
% Fields: VERSION, DATE, DESCRIPTION, OWNER NAME.
% See LPM-51 for version number policy.
\setDocChangeRecord{%
  \addtohist{1}{2022-XX-XX}{Unreleased.}{Leanne Guy}
}


\begin{document}

% Create the title page.
\maketitle
% Frequently for a technote we do not want a title page  uncomment this to remove the title page and changelog.
% use \mkshorttitle to remove the extra pages

% Content
\section{Introduction}\label{sec:into}

% Leanne - introduce subject  and associated with requirements 
\section{Use Cases}\label{sec:usecases}

Here we present the motivating use cases for the measurement of sub-threshold DIASources.

\subsection{Quality Assessment}\label{ssec:qa}
% Eric 

\subsection{Strong Lensing}\label{ssec:sl}
% Leanne

\subsection{Target of Opportunity}\label{ssec:too}
% Eric

\subsection{Potentially Hazardous Asteroids}\label{ssec:pha}
% Mario
\section{Technical Consideration}\label{sec:technical}
% Eric
\section{Community Proposal Process}\label{sec:process}
% Melissa

\textbf{Timeline --}
As the distribution of subthreshold alerts is planned to begin in year 2 of the LSST, the first community proposals will be solicited during the first half of year 1.
The first deadline will be timed to match the midpoint of year 1 (i.e., the observational cutoff date for Data Release 1).
Thereafter, proposals would be solicited on a yearly basis.

\textbf{Proposals --}
The community will be asked to provide proposals for subthreshold alerts that includes three components:

\begin{enumerate}
\item The estimated science impact, and motivation regarding why the science goals require alerts within 60 seconds and cannot be achieved using the Prompt Products Database, which is updated within 24 hours.
\item A detailed description of how the subthreshold alerts would be identified, e.g., lists of coordinates with inclusion radii, sky bounding box, or other difference-image source detection or measurement parameters.
\item A consideration of the technical considerations that speaks to the feasibility of the proposed subthreshold alerts.
\end{enumerate}

In keeping with best practices for research inclusion, the proposals will be anonymized (e.g., using the NOIRLab TAC system or similar).

As alerts are the only fully public LSST data product, the proposals will also be public.

A more detailed call for proposals and a template will be provided at the start of LSST Operations.

\textbf{Evaluation --} A technical review for proposal feasibility will be done first, by a small committee within Rubin Data Production team, headed by the lead of Alert Production.
If, after this technical review, the requests for subthreshold alerts are oversubscribed, the Rubin Observatory Director will assemble a 5 to 6 person scientific review committee.
This committee will be composed of scientists (Rubin staff and/or science community volunteers) with expertise that spans extragalactic and Galactic transients and variables and Solar System objects.
The committee will evaluate the proposals' scientific impact and urgency, write summary reports for each proposal, and provide a prioritized list for implementation to the Director.

\textbf{Report --} The Director will combine the technical and scientific reviews and publicly post a report on which proposals will be implemented.
This report will include explanations for rejected proposals so that proposers may learn and make revisions for the next round.

\section{Timeline}\label{sec:timeline}
% Leanne


\appendix
% Include all the relevant bib files.
% https://lsst-texmf.lsst.io/lsstdoc.html#bibliographies
\section{References} \label{sec:bib}
\renewcommand{\refname}{} % Suppress default Bibliography section
\bibliography{local,lsst,lsst-dm,refs_ads,refs,books}

% Make sure lsst-texmf/bin/generateAcronyms.py is in your path
\section{Acronyms} \label{sec:acronyms}
\input{acronyms.tex}
% If you want glossary uncomment below -- comment out the two lines above
%\printglossaries





\end{document}
